\chapter{Inferring the location of RTT changes}
\label{sec:infer}
\section*{Abstract}

\section{Perform measurement-based TE without direct measurements}
Locating the cause of RTT changes is an intriguing research topic in its own right.
Meanwhile, it is as well beneficial for measurement-based TE when measurements toward certain destination prefixes are not available.

\marginpar{How the RTT toward a destination prefix is measured?}
As explained in Section~\ref{sec:intro}, in order to measure the path performance toward a destination prefix, we need to, in first place, identify some hosts in that prefix that can be measured, or respond to our measurements.
One possible approach identifying them is to look for hosts listening on some common TCP ports, e.g. 80, 443 etc., in the traffic exchanged with that destination prefix or through port scanning.
\marginpar{Several ways exists to actively measure the RTT through TCP. Some methods with few footprint and low measurement are employed in port scanning~\cite{nmap}, like SYN (also known as half open) or FIN stealth.}
Then RTT toward the destination prefix is represented by the measurements toward identified hosts.

\marginpar{Selected prefixes: destination prefixes with important volumn thus selected for TE, more detail in Section~\ref{sec:pref_selec}.}
However, not all selected destination prefixes have such hosts with open ports, and
consequently leaves measurement-based TE without measurements.
Take client SA appeared in Section~\ref{sec:pref_selec} as an example, on average 15\% of its outbound traffic involving $\sim 330$ destination prefixes are without measurements.
More than 70\% of the `un-measured' traffic flows toward prefixes owned by mobile operators during peak hours.
In the foreseeable future, the proportion of such traffic would be even bigger, given the overall tendency of increasing usage on mobile devices.
At this point, we face with the problem of \textit{how to perform measurement-based TE without direct measurement toward the destination}. 

One possible approximation is to group `un-measurable' prefixes with those have measurements based on topological/geographical locality, for example those belonging to same AS or city. The problem with it is two-fold. First, locality is not necessary a good enough heuristics for shared fate in RTT performance. 
Assuming geographical locality, it at most provides an rough estimation on the baseline RTT in reaching the destination, not regarding issues with the IP geolocation precision~\cite{Poese2011}. It however does not indicate similarity in paths, nor offer enough distinction on instant transmission performance among multiple available paths.
Assuming topological locality, say AS-level, is more relevant, but still not good enough.
As it has been pointed out, AS is not an atomic point~\cite{Muhlbauer2006}. Sub-AS structures can lead to difference in path chosen by others ASes in reaching its different prefixes. A recent study offers a close look on the heterogeneity of sub-AS routing behavior~\cite{Lee2016}. Finally, such approximation is helpless when the entire AS is not `measurable'.
Second, in order to prove the effectiveness of such heuristics, one needs eventually measurements and the results could hence be biased toward/over-fitted for the part of Internet with measurements. Assuming the same for the rest without measurement data would be unconvincing.

Still, existing measurements, though toward destination other than the ones lack of them, can be taken advantage of. The idea is to locate the causes for important RTT changes. If the causes, can be an AS or an inter-AS link, sit on the BGP path toward the destination prefixes without measurements, we reasonably assume that those `un-measured' prefixes shall experience similar RTT changes, and triggers further route re-selection TE process.

If given measurements on a single path that undergo RTT changes, it is impossible to tell which part of the path causes the changes. However, if multiple paths with common parts and divergent parts are measured, it is then possible to infer for certain RTT changes, which part of the paths causes the change. We explain the inference logic in this chapter along with a working demo system.

\section{Relationship to network delay tomography}
It is not difficult to spot that RTT change cause inference is somewhat similar to the quest of network delay tomography~\cite{Coates2002}, i.e to infer the internal link delay characteristics using end-to-end measurements. The similarity not only lies in the formulation of the problem, but as well as in the assumptions and general idea of inference.

Many tomography works assume that measurements toward difference destinations experience similar delay on the shared links. As a matter of fact, these works either use multicast~\cite{LoPresti2002} or closely time spaced unicast measurements\cite{Shih2003,Tsang2003} to ensure this assumption. In our case, this intuitive assumption can be naturally extended: multiple RTT measurements shall undergo a same RTT change if caused by the common part on their paths.

The above assumption serves for inference. In delay tomography, since the common links contribute equally to end-to-end delay measurements, then the difference in end-to-end measurements can only come from the divergent part of the end-to-end path. With carefully designed measurement sets, it is then possible to infer for each internal link its delay properties~\cite{Lawrence2006}.
The basic inference logic for RTT change cause inference follows the same spirit, yet in slightly different form. That is, if multiple end-to-end measurements with common and divergent part in their paths experience at the same time a significant RTT change, then the cause for this change probably comes from the common part.

Despite the similarity between the two problems, the fundamental outcome wanted at the end of inference differs.
In delay tomography, one wishes to reconstruct the the probability distribution of delay on internal links, where the likelihood of each end-to-end delay measurements is parameterized by a convolution of internal links' delay distribution, from source to destination.
The parameters for internal link delay distribution can then be solved through maximum likelihood estimation of the end-to-end path delay distributions given the end-to-end measurements.
For that purpose, a series of methods are developed either to accelerate the maximum-likelihood estimation of delay distribution~\cite{Liang2003, Tsang2003}, or to capture the time-varying nature of link delays~\cite{Shih2003,Coates2002a,Tsang2003}.
If we were to apply the same approach to our RTT measurements after changepoint detection (Section~\ref{sec:cpt_rtt}), we might arrive at a probability distribution on the likelihood of a link causing significant RTT change over the period of measurements. However, it doesn't tell at what exact moment the link caused a significant RTT change.

Further, one might wonder whether it is possible to apply delay tomography methods directly to internet RTT measurements over a relatively short time range. And then detect whether the delay distribution experiences obvious change over time, or exhibit multimodality for certain link.
This approach is however not the most appropriate for various reasons. First, the assumption of
same delay contribution from same link to different end-to-end measurements no longer holds for Internet RTT measurements for TE uses. It is because the timing of different measurements are not strictly synchronized. Rather, random factors are deliberately added to each individual measurement in building the real system to avoid creating periodic peaks of measurement traffic that might introduce interference to each other and hence harm measurement reliability. Moreover,  RIPE Atlas measurements (Section~\ref{sec:ripe_atlas}) wildly employed in this work do not guarantee strict synchronization among different measurements.
Those measurements are individually performed by probes loosely coordinated. The timing depends basically on the moment when each probe starts for the first time.
Second, delay tomography introduces potentially a scalability issue for Internet measurements, as it is first conceived for intradomain uses and are in most cases validated solely on simulated networks way much smaller than the actually part of the Internet over which the traffic of a typical content/hosting/service provider could span.

\section{Change inference from changepoint detection}
\label{sec:inference}
Through the case study in Section~\ref{sec:ripe_case_study}, we realized that certain RTT changes observed on one path can be potential shared by other paths as well.
It implies that there is a potential common cause for such RTT change experience by multiple non-identical paths.
Probably, the cause lies in the common part of these impacted paths.
Realizing that it is feasible quest, we set out developing inference logic that narrows down the cause to a single AS or inter-AS link when possible.