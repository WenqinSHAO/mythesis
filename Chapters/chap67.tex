\chapter{Conclusion}

\section{Thesis Summary}

This thesis is developed around a central pursuit of \textit{making better use of various network measurements to improving the transmission performance for stub ASes from the outbound perspective.}
We came up with methods that predictively select destination prefixes associated with large volume through studies on the temporal dynamism of volume measurements.
These methods allow traffic engineering operations to be focused only on these important targets while still cover a majority of traffic, hence improving the scalability of traffic engineering system.
Later on, we focus on latency measurements. 
We showcased and diagnosed some data quality issues previously unattended.
Guidelines to mitigate their impacts on data processing and route selection were discussed.
To better interpret performance measurements, we introduced changepoint analysis to the processing of RTT measurements, so that significant changes in path performance can be systematically detected and trigger route re-selection.
We built an evaluation framework to quantify the robustness and sensitivity of detection methods on typical RTT measurements.
Further, with the time series simplification by changepoint analysis, we grouped together RTT measurements that experience a same RTT change and inferred the network location that potentially causes that change. This visibility enables route optimization for prefixes that we were not able to measure directly.

\section{Contributions}

\section{Scalable prefix selection}
It is recommended to perform traffic engineering only for destinations with import traffic volume. One key challenge is to efficiently predicting the volume importance for a great amount of destination prefixes.

We analyzed real traffic measurements from nine different networks located in five different countries to acquire a realistic view on the distribution of traffic volume associated with BGP prefixes, as well as its variation in time.   
We observed that the prefixes with most important accumulated volume over a week have relatively stable hourly volume. 
Based on this observation, we proposed three simple 
metrics (also easy to compute) to proactively select prefixes with important foreseeable traffic volume.
We demonstrated that the metrics we proposed lead to better volume coverage compared to the existing solutions, the Grey model.
%Furthermore, we evaluated the transmission performance for selected destination prefixes using different transit providers. We simulated as well a dynamic route decision algorithm. 
%The results showed that with even a fairly basic mechanism, the overall RTT performance could be improved by $20\%$ compared to the best available transit provider in some networks studied. 

\section{Data quality concerns}
We studies two data quality issues previously unattended. One comes from the RIPE Atlas measurement platform, the other is specific to path performance measurement with multiple probes.

The problem with Atlas was that some data points were missing for measurements meant to be performed regularly.
We investigated all the available RIPE Atlas probes back then over one month. Only 60\% of v3 probes have complete measurement length. 
$2/3$ of missing segments occurred while probes remain connected. 
Half of these segments are no more than 2 measurements in length, and are thus likely to be caused by scheduling issues. However, around $25\%$ of them lasts long($\geq 1h$). 
To help advance the investigation, we share with the RIPE Atlas team all the long missing segments identified.

When measuring the performance of an AS path, multiple hosts/Atlas probes in the same destination prefix can be employed. We questioned the difference among these measurements, why it exists and how it could impact route selection.
We found out that RTT time series collected in this study demonstrate diverse variation shapes though one common AS path is measured.
We clustered these RTT time series to reveal their group structure by extracting several features as their data representation. 
Resulted clusters successfully separate noisy traces from smooth ones according to human intuition and expertise.
It confirmed that RTT measurements need to be ``cleaned'' before traffic engineering usages.
Furthermore, we located the occurring location of most variations in end-to-end RTT measurements by applying the presented clustering methods to the first hops of traceroute measurements.
The results showed that most variations come from the access network in this specific case.

\section{Change detection for RTT measurements}

In this chapter, we proposed an evaluation framework for change detection on RTT time series.
The framework is robust with human-labelled dataset and weights RTT changes according to their importance in network operation. We further designed a data transformation adapted to RTT measurements to improve the detection sensitivity of some detection methods.
In detecting path changes, we distinguish those caused by routing changes from those due to load balancing.
Finally, we correlate the detected RTT and path changes by establishing an one-to-one matching between them. 
We investigated the sensitivity distinction across different change detection methods. 
Hidden issues with path changes are as well revealed.

This work is mere a facilitator for measurement-based TE. Further efforts are required in building a working system. To name a few: online detection of RTT changes, route selection logic triggered by change detection  etc.

\section{Change location inference}

\chapter{Future works}
In order to further improve prefix selection methods, we have shown that capturing bursty prefixes is the key. 
To this end, we could group prefixes by their activity profiles. 
For each group, selection method is adapted to its traffic dynamism.
When dealing with prefixes with regular volume patterns and small hourly variation, the simple metrics proposed in this work perform already sufficiently well. 
Nevertheless, for bursty prefixes, we might need a more sophisticate model that extracts additional activity features, for instance long term periodicity.
The burstiness index $\beta$ proposed in this chapter, shown to be very expressive, could be potentially used in prefix characterization and classification and thus is worthy of future work.
