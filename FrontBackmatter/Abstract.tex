%*******************************************************
% Abstract
%*******************************************************
%\renewcommand{\abstractname}{Abstract}
\pdfbookmark[1]{Abstract}{Abstract}
\begingroup
\let\clearpage\relax
\let\cleardoublepage\relax
\let\cleardoublepage\relax

\chapter*{Abstract}

As transit price continues to drop, mulithoming has now become a common practice among many medium and small size networks. Yet, how to improve the transmission performance through wise employment of multiple Internet paths remains challenging. One major roadblock is the performance agnostic Internet routing protocol, \acf{BGP}, that is not going to be obsolete shortly.

Both previous studies and commercial solutions suggest regularly measuring the performance on all available paths in reaching destination networks.
Then the best performing path is selected based on these measurements, i.e. measurement-based \acf{TE}.
However, plenty issues still exist and left open in building a satisfying TE system of this kind.
We tackle some of the most pronounced problems in this dissertation to bring improvements in these aspects: measurement scalability, interpretation of performance data and visibility on causes of performance changes. 

We first study the methods that predictively select destination prefixes associated with large traffic volume.
These methods allow traffic engineering operations to be focused on only a few important targets while still covering a majority of traffic, hence improving the scalability of the whole system.

Later on, we focus on latency measurements. 
We showcase and diagnose some data quality issues previously unattended.
Guidelines to mitigate their impacts on data processing and route selection are discussed.

To better interpret performance measurements, we introduce changepoint analysis to the processing of \acf{RTT} measurements, so that significant changes in path performance can be systematically detected and trigger route re-selection. 
To enable further work in this direction, we devise an evaluation framework quantifying the robustness and sensitivity of detection methods on typical RTT time series.

Further, we propose to infer the location of performance changes seen in end-to-end measurements. This visibility enables route optimization for prefixes that can not be measured directly.
We develop inference procedures for \acf{AS} and inter-AS links based on two intuitive assumptions.
To better illustrate the inference process and the identified causes for RTT change, we devise two interactive visualization tools to plot results on a topology graph learned through path measurements.
\vfill


\iffalse
\begin{otherlanguage}{french}
\pdfbookmark[1]{Résumé}{Résumé}
\chapter*{Résumé}
Le résume en français...
\end{otherlanguage}
\fi

\endgroup			

%\vfill
